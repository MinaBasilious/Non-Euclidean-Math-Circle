\documentclass{article}
\usepackage{graphicx} % Required for inserting images
\usepackage{amsfonts,amsmath,amssymb,amsthm}

\title{Answers Sequence \& Series}
\author{Mina Basilious}
\date{July 2023}

\begin{document}

\maketitle

\section*{Answers}
\subsection*{\# 1}
for this simple geometric series we just substitute it in $\dfrac{a_\circ}{1-r}$.\\
$$\text{For } |x|<1, \ 1+x+x^2+x^3+x^4+...=\boxed{\frac{1}{1-x}}$$
\subsection*{\# 2}
$$\sum_{k=0}^n \log(x^k)$$
$$\log(1)+\log(x)+\log(x^2)+...+\log(x^n)$$
Knowing that $\log(ab)=\log(a)+\log(b)$ then
$$\log(1\cdot x \cdot x^2 \cdot ... \cdot x^n)$$
From the properties of the exponents, we could add them with multiplications.
$$\log(x^{0+1+2+3+...+n})$$
$$\log(x^{\sum_{i=0}^ni})$$
And from the properties of logarithms, we could bring the inner exponent out $\log(a^b)=b\log(a)$.
$$\sum_{i=0}^ni \cdot \log(x)$$
And $\sum_{i=0}^ni$ is simple arithmetic series that has generalized formula as $\frac{n(n+1)}{2}$.
$$\sum_{k=0}^n \log(x^k)=\boxed{\frac{n(n+1)\log(x)}{2}}$$
\subsection*{\# 3}
We will proof by induction.
We check the initial condition firstly
$$a_0=2^0=1 \ \checkmark$$
Then we assume the statement is true for $n=k$ and then show it holds for $n=k+1$, which will prove for all terms of the sequence.
$$2^k = a_k$$
As we know from
$$2^{k+1}=2 \cdot 2^k = 2a_k$$\\
Then\\
$$a_{n+1}=2a_n \implies a_n = 2^n$$

\subsection*{\# 4}
Leave it for you to figure it out.\\
If you could not reach any solutions read ``Proofs That Really Count" at the beginning of Chapter 8 Number Theory.

\end{document}
